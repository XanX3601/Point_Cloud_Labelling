\section{Introduction}
\label{sec:intro}

The segmentation and labelisation of point cloud is a challenging computational problem that could be one key problem in the future of Artificial Intelligence (AI). Being able to interpret an environment read through technologies like \gls{lidar} gives the ability to autonomous system to understand and navigate in their surroundings. The autonomous vehicles are a perfect example of such use and a lot of constructor bet on this type of technology.\\

Previous approach on the problem required previous knowledge on the point cloud such as normals, roundness, etc. In 2016, \citeauthor*{7900038} \cite{7900038} proposed a new approach of the problem using deep learning similar to the ones used on images. The novelty of their approach reside in the fact that no prior information on the point cloud is required. Their algorithm is split into three phases, it firsts computes the voxelization of the point cloud in order to obtain well defined grid. These grids are passed to a \gls{3dcnn} that classify them into the different categories studied.
Finally, the labels of each points is deduced from the output of the network.\\

In this article, we propose to test their approach and try it on more complex data. Our work can be found \href{https://github.com/XanX3601/point_cloud_labeling.git}{here}. The goal is to see if their approach holds on with more categories. The rest of this paper is organized as follow. In section~\ref{sec:summary} we summarize the approach \cite{7900038}. In section~\ref{sec:dataset} we describe our choosen dataset. In section~\ref{sec:implementation} we discuss our choices and problems in the implementation of the algorithm. In section~\ref{sec:results} we reveal the results of our experiments on a new dataset of point cloud. In section~\ref{sec:future} we expose some ideas for future work that may lead to an improvement of the global performance. In section~\ref{sec:conclusion} we conclude this article.
